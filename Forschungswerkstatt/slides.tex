% Compile using LuaLaTeX!

\documentclass{beamer}

% Beamer
\usetheme{Warsaw} % or split
\usecolortheme{grass}
\setbeamercovered{transparent}
% Navigation ausblenden
\setbeamertemplate{navigation symbols}{}%remove navigation symbols

% Zeilenabstand verringer
\renewcommand{\baselinestretch}{0.9}

% more flexible than babel. Requires LuaLaTeX or XeTeX
\usepackage{polyglossia}
\setdefaultlanguage{german}

% LuaTeX and XeTeX (but LuaLaTeX is cooler) -> fontspec
\usepackage{fontspec}

% extra symbols
\usepackage{amsmath}
\usepackage{amsfonts}
\usepackage{amssymb}
\usepackage{amsthm}
\usepackage{multimedia}
\usepackage{hyperref}
\usepackage[]{graphicx}
\graphicspath{{../resources/}}
\usepackage{listings}
\usepackage{pdfpages}
\usepackage{color}

\author{FOSS-AG}
\title{Vorstellung}

\begin{document}	
	\begin{frame}
		\vspace{0.2cm}
		\begin{minipage}{0.56\linewidth}
			\begin{itemize}
				\item wöchentliche Treffen
				\item Veranstaltung rund um die Themen Linux und Free-Software
				\begin{itemize}
					\item Installpartys
					\item Themenabende
					\item Workshops
					\item Hack'n'Snack
				\end{itemize}
				\item Kooperationen mit lokalen Hackspaces
				\item Weitere Infos: \textbf{www.foss-ag.de}
			\end{itemize}
		\end{minipage}
	
		\hspace*{0.8\linewidth}
		\begin{minipage}{0.35\linewidth}
			\includegraphics[scale=0.2]{tux}
		\end{minipage}
	\end{frame}
\end{document}